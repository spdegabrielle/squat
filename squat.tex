\input squat-mac

\title{On adding quaternions to Scheme}

\ifx\shipout\undefined
\centerline{\urlh{squat.tar.gz}{Download}}
\fi

\centerline{\urlh{http://www.cs.rice.edu/~dorai}{Dorai Sitaram}}

\bigskip

\n The numbers supported by the Scheme programming
language form a tower, where the set of numbers at each
level is a subset of the set at the level above.  The
levels are, respectively: integers, rationals,
reals,\f{In practice, an implementation does not
differentiate between rationals and reals.}
and complex numbers.
This article investigates the potential extension of
Scheme's numeric tower upward to include {\em
quaternions}
\cite{onquat}, a set of hypercomplex numbers that
properly includes the complex numbers, and by
extension, all the numbers below.

\section{Quaternions}

A quaternion $q$ is a 4-tuple $(w,x,y,z)$, where $w$,
$x$, $y$, and $z$ are real.  To give it the aspect of a
number, it is written $w + ix + jy + kz$, where $i$,
$j$, and $k$ are {\em imaginary units} that satisfy the
conditions $ii = jj = kk = -1$, $ij = -ji = k$, $jk =
-kj = i$, and $ki = -ik = j$.\f{Invoking the
associativity of multiplication, these conditions can
be more succinctly stated as $ii = jj = kk = ijk =
-1$.}   Clearly, multiplication on quaternions does not
commute.  $w$ is $q$'s real (or scalar) part, and
$x$, $y$, and $z$ are  its $i$-, $j$-, and
$k$-part respectively.  $ix + jy + kz$ is 
$q$'s vector part.  A quaternion with zero $j$- and
$k$-parts is an ordinary complex number, and if,
further, its $i$-part is also zero, it is just a real
number.  A quaternion with zero real part is called a
vector quaternion or, simply, vector.\f{These are
indeed the vectors familiarly used in physics to
represent 3-dimensional quantities.  They are not be confused
with Scheme's one-dimensional array data type called \q{vector}.} 

Notation:  We will use $q_n$ (where $n$ is either an
integer or absent) interchangeably with
$w_n+ix_n+jy_n+kz_n$.  $q_n$'s vector part
$ix_n+jy_n+kz_n$ is also written $v_n$ and its unit
vector (see below) is written $u_n$.  \q{|meta[E] =>
|meta[e]} means that the Scheme expression $E$
evaluates to a Scheme value $e$.  In order to prevent
confusion, we add a subscript $q$ to the name of a
Scheme procedure that has been made quaternion-aware.
For example, \q{exp} is the exponential procedure
described in R5RS~\cite{r5rs}; whereas \q{exp|Q}
is the same procedure that has been extended to accept
a quaternion argument.

\section{Lexical representation of quaternions}

Recall that Scheme complex numbers that are not also
real are represented in {\em rectangular} (or cartesian) notation as 
\q{|meta[a]+|meta[b]i} or
\q{|meta[a]-|meta[b]i}, where $a$ is optional (for
numbers that are purely imaginary) or real, and $b$ is
a non-negative real.  Examples: 

\q{
+i
-i
+2i
-2i
3+4i
}

\n (The sign is never
optional, eg, $i$ must be represented as \p{+i} rather
than \p{i}, which latter is parsed as an identifier.)
We can naturally extend this to 
quaternions: thus $q$ above is written
\q{|meta[w]+|meta[x]i+|meta[y]j+|meta[z]k}.
The pluses can be
minuses, and any part that is zero may be omitted, with
the proviso that a vector quaternion must always start
with a sign or an explicit zero real part, in order to prevent being mistaken for an
identifier.  Examples of some Scheme quaternions that
are not also complex numbers: 

\q{
+j
+3k
-2+3i-3k
-6i-6k
3-4i+5j+6k
}

\section{Polar notation}

Scheme also permits users to enter complex numbers in a
notation that uses {\em polar} coordinates, viz, 
$\mu${\tt @}$\theta$, where $\mu$ and $\theta$ are real.
A corresponding notation for quaternions 
is also possible.  For example,
$q = w + ix + jy + kz$ has the polar (or spherical or
curvilinear) coordinates 
$\mu$, $\theta$, $\phi$, $\psi$, which are related to the
rectangular coordinates as follows:\f{The rectangular
coordinates can be deduced from the polar as
follows:
$$
\eqalign{
w &= \mu \cos \theta \cr
x &= \mu \sin \theta \cos \phi \cr
y &= \mu \sin \theta \sin \phi \cos \psi \cr
z &= \mu \sin \theta \sin \phi \sin \psi
}
$$}

$$
\eqalign{
\mu &= \sqrt{w^2 + x^2 + y^2 + z^2} \cr
\theta &= \tan^{-1} (\sqrt{x^2 + y^2 + z^2} / w)\cr
\phi &=  \tan^{-1} (\sqrt{y^2 + z^2} / x) \cr
\psi &= \tan^{-1} (z/y)
}
$$

\n $\mu$, $\theta$, $\phi$, $\psi$ are respectively the
{\em magnitude}, {\em
angle}, {\em colatitude}, and {\em longitude}
of $q$.\f{The more usual terms for {\em magnitude} and {\em angle}
are {\em modulus} and {\em amplitude} respectively, but 
we wish to remain compatible with Scheme's already established
names for complex numbers.} 
Extending Scheme's notation for complex numbers,
$q$ may be represented as
$\mu\,${\tt @}$\,\theta\,${\tt\%}$\,\phi\,${\tt\&}$\,\psi$.  Any zero angle may
be omitted along with its prefix.  If the
colatitude and longitude are both zero, the number is
complex. 

As for complex numbers, polar notation
is used only to {\em enter} quaternions.  Scheme outputs
quaternion values exclusively in rectangular.  Thus,

\q{
-3@4%-5&6
|evalsto
1.960930862590836+0.6440287493492101i+2.0904336369493484-0.6083291310311992k
}

\section{Building quaternions and taking them apart}

The predicate \q{quaternion?} checks if its argument is
a quaternion.  Because quaternions now form the pinnacle of Scheme's
numeric tower, \q{quaternion?} returns true (\q{#t}) for any
number, and thus agrees with Scheme's \q{number?} predicate.

The Scheme complex-number procedures \q{real-part},
\q{imag-part}, \q{magnitude}, and \q{angle} are
extended to accept quaternion arguments.  They return,
respectively, the real part, the $i$-part, the magnitude,
and the angle, of their argument.  To these are
added four new related procedures: \q{jmag-part},
\q{kmag-part}, \q{colatitude}, and \q{longitude}
return, respectively, the $j$-part, $k$-part, colatitude, and
longitude of their argument.

The procedure \q{vector-part} returns the vector part of its
argument.

The procedures \q{make-rectangular} and
\q{make-polar} are extended to take two extra arguments
in order to produce quaternions that aren't just complex numbers.

\q{
(make-rectangular|Q 3 4 5 6)
|evalsto 3+4i+5j+6k

(make-polar|Q -3 4 -5 6)
|evalsto
1.960930862590836+0.6440287493492101i+2.0904336369493484-0.6083291310311992k
}

\n Note that the result {\em printed} by Scheme is the rectangular
equivalent of the polar coordinates specified.  

Every quaternion $q$ has a special vector quaternion $u$
called its {\em unit vector}, which is defined as

$$
{{ix + jy + kz} \over
\sqrt{x^2+y^2+z^2}
}
$$

\n In Scheme,

\q{
(define unit-vector
  (lambda (q)
    (*|Q (vector-part q)
       (/ (magnitude|Q q)))))
}

\n The magnitude of a unit vector is of course unity.


\section{Arithmetic}

Scheme numerical procedures that work on complex
numbers can be naturally extended to accept quaternion
arguments.\f{Procedures like \q{<}, \q{>}, \q{<=},
\q{>=}, \q{positive?}, \q{negative?}, \q{min},
\q{max}, \q{floor}, \q{ceiling}, \q{truncate},
\q{round}, \q{rationalize} two-argument \q{atan},
\q{make-rectangular} and \q{make-polar} are not defined
for quaternions, but they are not defined for complex
numbers either.}

The procedures \q{=}, \q{+}, \q{-} extend in the expected manner for
quaternion arguments ---
they all operate on the parts severally.

\q{
(=|Q 3+4j 3+4j)      => #t
(=|Q 3+4j 3+4j 3+4k) => #f
(+|Q 3+4j 3+4j)      => 6+8j
(+|Q 3+4j 3+4j 3+4k) => 9+8j+4k
(-|Q 4+5j 2-3k)      => 2+8k
(-|Q 2+3j-4k)        => -2-3j+4k
}

\n The multiplication procedure \q{*|Q} (using the identities 
$ii = jj = kk = ijk = -1$) can be implemented as
follows: 

$$
\eqalign{
q_1 q_2 = &  \quad 
   (w_1 w_2 - x_1 x_2 - y_1 y_2 - z_1 z_2)  \cr
&+ (w_1 x_2 + x_1 w_2 + y_1 z_2 - y_2 z_1)i  \cr
&+ (w_1 y_2 + y_1 w_2 + z_1 x_2 - z_2 x_1)j \cr
&+ (w_1 z_2 + z_1 w_2 + x_1 y_2 - x_2 y_1)k
}
$$

\n 
Example: 

\q{
(*|Q 4-3i+2j+6k 4+5i-7j+9k)
|evalsto -9+68i+37j+71k
}

\n Generally, multiplication is not commutative.
However if two quaternions have the same {\em unit
vector} disregarding sign, they do commute on multiplication, and
furthermore, their product has the same unit vector.
The square of a unit vector (its product with itself) is $-1$.

If $q_1$, $q_2$ are vector quaternions, ie,
$w_1 = w_2 = 0$, then the real part of
$q_1 q_2$ is the negative of the {\em dot product} $q_1 \cdot q_2$
and the vector part of $q_1 q_2$ is the {\em cross product} $q_1
\times q_2$.  

\q{
(define dot-product
  (lambda (q1 q2)
    (if (not (and (= (real-part|Q q1) 0)
                  (= (real-part|Q q2) 0)))
        (error 'dot-product "arguments must be vector quaternions")
        (- (real-part|Q (*|Q q1 q2))))))

(define cross-product
  (lambda (q1 q2)
    (if (not (and (= (real-part|Q q1) 0)
                  (= (real-part|Q q2) 0)))
        (error 'cross-product "arguments must be vector quaternions")
        (vector-part (*|Q q1 q2)))))
}

\n If the imaginary parts of $q_1$ and $q_2$ are
severally the negatives of each other, then the
product $q_1 q_2$ is real.  Such $q_1$, $q_2$ are
termed {\em conjugates} of each other.  They have the
same magnitude, and their product is the square of that
magnitude.

\q{
(define conjugate
  (lambda (q)
    (make-rectangular|Q
      (real-part|Q q)
      (- (imag-part|Q q))
      (- (jmag-part q))
      (- (kmag-part q)))))
}

\n We can use fact (1) to implement the {\em reciprocal} of a
quaternion.  Multiply both numerator and 
denominator of $1/q$  by $q$'s conjugate.  Unary \q{/}
is the Scheme procedure for reciprocals.  Example:

\q{
(/|Q 8+3i+4j+5k)
|evalsto 
0.07017543859649122-0.02631578947368421i-0.03508771929824561j-0.043859649122807015k
}

Multiplication by the reciprocal leads to a definition
of division.  Dividing $q_1$ by $q_2$ is viewed as
multiplying $q_1$ by $q_2^{-1}$.
The noncommutativity of
multiplication allows a choice between putting the
reciprocal before or after $q_1$.  Thus division from
the left and from the right are different.  

In keeping with the visual layout of Scheme's \q{/}
procedure, we can define 
\q{(/|Q q1 q2 ...)} with 
\q{(*|Q q1 (/|Q q2) ...)}, but to prevent confusion,
it is best to always use
explicit reciprocals (unary \q{/|Q}) and multiplication
(\q{*|Q}).

\section{Exponential and logarithmic functions}

As for complex numbers, we use the Maclaurin series to
define quaternion extensions of the transcendental
functions.
Thus, for quaternion $q = w + uv$, where $uq$ is its unit
vector and $w$ and $v$ are real, we can derive

$$
e^q = e^{w + u v} =  e^w(\cos v + u \sin v)
$$

\n $u$ functions much the same as $i$ for complex numbers.
The difference from the complex case is that 
$u$ depends on (ie, varies with) $q$, whereas $i$ is the same for all
complex $z$.

In Scheme,

\q{
(define exp|Q
  (lambda (q)
    (let ((w (real-part|Q q))
          (u (unit-vector q))
          (v (magnitude|Q (vector-part q))))
      (*|Q (exp w)
         (+|Q (cos v) (*|Q u (sn v)))))))
}

\n Note that the unit vectors of $q$ and $e^q$
are the same, disregarding sign.  The inverse of
of the exponential function is the natural logarithm,\f{Following Scheme
practice, we use $\log$ to mean $\log_e$ rather than
$\log_{10}$.} and is
given by:


$$
\log (w+uv) = {1\over 2}\log(w^2+v^2) + u \tan^{-1} (v/w)
$$

\n (where the $\log$ on the right only need be defined 
for positive reals).  To make the $\log$ function single-valued,
we choose $\tan^{-1} (v/w)$ between $-\pi$ adn $\pi$.  In Scheme,

\q{
(define log|Q
  (lambda (q)
    (let ((w (real-part|Q q))
          (u (unit-vector q))
          (v (magnitude|Q (vector-part q))))
      (+|Q (*|Q 1/2 (log (magnitude|Q q)))
         (*|Q u (atan v w))))))
}

\n We can use \q{log|Q} to define \q{expt|Q}, since:
$$
q_1 ^{q_2} = (e ^ {\log q_1})^{q_2} = e ^{(\log q_1)q_2}
$$

\n In Scheme,

\q{
(define expt|Q
  (lambda (q1 q2)
    (exp|Q (*|Q (log|Q q1) q2))))
}
 
\n The \q{sqrt|Q} function then is:

\q{
(define sqrt|Q
  (lambda (q)
    (expt|Q q 1/2)))
}

\section{Trigonometric functions}

For $\sin q$ and $\cos q$,  we have
(using the Maclaurin's series for $\sinh$ and $\cosh$): 

$$
\eqalign{
\sin(w + uv) &= \sin w \cosh v + u \cos w \sinh v\cr
\cos(w + uv) &= \cos w \cosh v - u \sin w \sinh v
}
$$

\n In Scheme,\f{\q{sinh} and \q{cosh} for reals are not
part of the Scheme standard~\cite{r5rs} but they
can be readily defined using the
identities 
$$
\eqalign{
\sinh x &= (e^x - e^{-x})/2 \cr 
\cosh x &= (e^x + e^{-x})/2 \cr
}
$$}

\q{
(define sin|Q
  (lambda (q)
    (let ((w (real-part|Q q)) 
          (u (unit-vector q))
          (v (magnitude|Q (vector-part q))))
      (+|Q (*|Q (sin w) (cosh v))
         (*|Q u (cos w) (sinh v))))))

(define cos|Q
  (lambda (q)
    (let ((w (real-part|Q q)) 
          (u (unit-vector q))
          (v (magnitude|Q (vector-part q))))
      (-|Q (*|Q (cos w) (cosh v))
         (*|Q u (sin w) (sinh v))))))
}

\n  $\tan q$ can be defined as $(\sin q)/(\cos q)$.
Since $\sin q$, and $\cos q$ have the same unit vector
as $q$, the direction of the division doesn't
matter.  

\section{Inverse trigonometric functions}

The following are derived 
using Maclaurin's and straightforward quadratic solving:

$$
\eqalign{
\sin^{-1} q &= - u \log (uq + \sqrt{1 - q^2}) \cr
\cos^{-1} q &= - u \log (q + \sqrt{q^2 - 1}) \cr
\tan^{-1} q &= {u\over 2} \log {u+q \over u-q}
}
$$

\n In Scheme,

\q{
(define asin|Q
  (lambda (q)
    (let ((u (unit-vector q)))
      (-|Q (*|Q u (log|Q (+|Q (*|Q u q)
                              (sqrt|Q (-|Q 1 (*|Q q q))))))))))

(define acos|Q
  (lambda (q)
    (let ((u (unit-vector q)))
      (-|Q (*|Q u (log|Q (+|Q q 
                              (sqrt|Q (-|Q (*|Q q q) 1)))))))))

(define atan|Q
  (lambda (q)
    (let ((u (unit-vector q)))
      (*|Q 1/2 u (log|Q (*|Q (+|Q u q) (/|Q (-|Q u q))))))))
}
  
  
  
  
\section{Signatures}

The following collects the signatures of all the
quaternion procedures proposed above.  The types {\em
number} and {\em quaternion} are synonymous.  I shall
use the name {\em quaternion} merely to emphasize that
a function result can be a quaternion.  Arguments
enclosed in brackets are optional.

{\obeylines
\q{quaternion?} : {\em any} \tu {\em boolean} 

\q{real-part|Q} : {\em number} \tu {\em real} 
\q{imag-part|Q} : {\em number} \tu {\em real}
\q{jmag-part} : {\em number} \tu {\em real} 
\q{kmag-part} : {\em number} \tu {\em real}

\q{vector-part} : {\em number} \tu {\em quaternion}
\q{unit-vector} : {\em number} \tu {\em quaternion}

\q{magnitude|Q} : {\em number} \tu {\em real}
\q{angle|Q} : {\em number} \tu {\em real}
\q{colatitude} : {\em number} \tu {\em real}
\q{longitude} : {\em number} \tu {\em real}

\q{make-rectangular|Q} : {\em real} \tymes {\em real} [\tymes {\em real} \tymes {\em real}] \tu {\em quaternion}
\q{make-polar|Q} :  {\em real} \tymes {\em real} [\tymes {\em real} \tymes {\em real}] \tu {\em quaternion}

\q{+|Q} : {\em number}* \tu {\em quaternion}
\q{-|Q} : {\em number}+ \tu {\em quaternion}
\q{*|Q} : {\em number}* \tu {\em quaternion}
\q{/|Q} : {\em number}+ \tu {\em quaternion}

\q{dot-product} : {\em number} \tymes {\em number} \tu {\em real}
\q{cross-product} : {\em number} \tymes {\em number} \tu {\em quaternion}
\q{conjugate} : {\em number} \tu {\em quaternion}

\q{exp|Q} : {\em number} \tu {\em quaternion}
\q{log|Q} : {\em number} \tu {\em quaternion}
\q{expt|Q} : {\em number} \tymes {\em number} \tu {\em quaternion}
\q{sqrt|Q} : {\em number} \tu {\em quaternion}

\q{sin|Q} : {\em number} \tu {\em quaternion}
\q{cos|Q} : {\em number} \tu {\em quaternion}
\q{tan|Q} : {\em number} \tu {\em quaternion}

\q{asin|Q} : {\em number} \tu {\em quaternion}
\q{acos|Q} : {\em number} \tu {\em quaternion}
\q{atan|Q} : {\em number} \tu {\em quaternion}
\q{atan} : {\em real} \tymes  {\em real} \tu {\em real} (two arguments)

} 

\section{References}

\input squat-bib

\bye
